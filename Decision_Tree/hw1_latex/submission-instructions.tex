\section{Submission Instructions}

You will submit two components on Gradescope:

\begin{itemize}
    \item \textbf{Report}: Submit your report as a pdf, containing your answers to the questions in 
    Section~\ref{sec:decision-trees} and your findings/results from the coding sections to the 
    corresponding assignment in Gradescope.
    \item \textbf{Code}: If you're working in Python, submit your \texttt{cross\_validation.py}, 
    \texttt{model.py}, and \texttt{train.py} files to the corresponding assignment in Gradescope.
    If you're not using python, submit your executable files following the instructions 
    below.
\end{itemize}


\begin{non-python-note}
    If you're not using Python, we ask that you make your code CLI executable from the CADE machines. 
    Please also add a REAMDE.md to your submission with instructions on how to run your code.

    \vspace{1em}

    Specifically, you need to have an executable file that trains and evaluates both the majority baseline classifier and your decision tree.
    It should print the train and test accuracies (the same values your include in your report) to the console.
    It should accept the following command-line flags:
    \begin{compactitem}
        \item \texttt{-t TRAIN\_PATH}: path to train csv file
        \item \texttt{-e EVAL\_PATH}: path to test csv file
        \item \texttt{-m MODEL\_TYPE}: takes one of \texttt{["majority\_baseline", "decision\_tree"]}
        \item \texttt{-d DEPTH\_LIMIT}: takes an integer and sets the depth-limit hyperparameter
        \item \texttt{-i IG\_CRITERTION}: takes one of \texttt{["entropy", "gini"]} (only needed if you're in CS 6350)
    \end{compactitem}

    \vspace{1em}

    Similarly, you need to have a different executable file that runs cross-validation on your decision tree, and accepts the flags \texttt{-c CV\_PATH} pointing to your \texttt{cv/} directory, 
    \texttt{-d DEPTH\_LIMIT} and \texttt{-i IG\_CRITERTION} (only needed if you're in CS 6350).
    It should print the best CV accuracy and corresponding hyperparameters (the same values your include in your report) to the console.

    \vspace{1em}

    We won't be able to assign partial credit for non-Python submissions if the code doesn't run.
    We also won't be able to provide debugging assistance during office hours.
    See the FAQ section on the class website for instructions on how to access CADE, and how to turn your code into a command-line executable.
\end{non-python-note}
